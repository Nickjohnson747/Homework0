\documentclass[11pt]{article}
% generated by Madoko, version 1.0.3
%mdk-data-line={1}


\usepackage[heading-base={2},section-num={False},bib-label={True}]{madoko2}


\begin{document}



%mdk-data-line={8}
\mdxtitleblockstart{}
%mdk-data-line={8}
\mdxtitle{\mdline{8}Homework 0}%mdk

%mdk-data-line={11}
\mdxtitlenote{\mdline{11}Due on Wednesday, September 7}%mdk
\mdxauthorstart{}
%mdk-data-line={16}
\mdxauthorname{\mdline{16}Nick Johnson}%mdk

%mdk-data-line={19}
\mdxauthoremail{\mdline{19}nj41@indiana.edu}%mdk
\mdxauthorend\mdtitleauthorrunning{}{}\mdxtitleblockend%mdk

%mdk-data-line={10}
\section{\mdline{10}1.\hspace*{0.5em}\mdline{10}Basic Text Typewriting}\label{sec-basic-text-typewriting}%mdk%mdk

%mdk-data-line={12}
\begin{itemize}[noitemsep,topsep=\mdcompacttopsep]%mdk

%mdk-data-line={12}
\item\mdline{12}Item 1%mdk

%mdk-data-line={13}
\item\mdline{13}Item 2%mdk

%mdk-data-line={14}
\item\mdline{14}Item 3%mdk

%mdk-data-line={15}
\item\mdline{15}Subscripting: Black\mdline{15}\mdsub{pit}\mdline{15}; Blue\mdline{15}\mdsup{angel}\mdline{15}%mdk
%mdk
\end{itemize}%mdk

%mdk-data-line={17}
\noindent\mdline{17}These are \mdline{17}\textbf{boldfaced words}\mdline{17}. Newly introduced notions will be emphasized in italic font such as \mdline{17}\emph{group theory}\mdline{17}. We could also add boldface to it, e.g., \mdline{17}\textbf{\emph{group theory}}\mdline{17}.%mdk

%mdk-data-line={19}
\mdline{19}Mathematics can be in-line like \mdline{19}$E=mc^2$\mdline{19}. Or, they can be elegantly presented in standalone blocks like below:%mdk
\label{euler-thm}%mdk
\noindent\mdline{21}\mdmathtag{(1)}\mdline{21}
\noindent\[%mdk-data-line={22}
  e = \lim_{n\rightarrow\infty} (1+\frac1n)^n
\]%mdk
\noindent\mdline{25}The above theorem can be referenced as equation\mdline{25}~\mdref{thm-euler}{(1)}\mdline{25}. We may prove it as a theorem.

%mdk-data-line={27}
\mdline{27}\textbf{Theorem 1}\mdline{27}. (\mdline{27}\emph{Euler's Theorem}\mdline{27})%mdk

%mdk-data-line={29}
\mdline{29}If \mdline{29}\emph{n}\mdline{29} and \mdline{29}\emph{a}\mdline{29} are positive integers and gcd(\mdline{29}\emph{n, a}\mdline{29}) = 1, then%mdk
\label{thm-euler}%mdk
\noindent\mdline{30}\mdmathtag{(2)}\mdline{30}
\noindent\[%mdk-data-line={31}
  a^{\phi(n)} \equiv1 \mod n
\]%mdk

%mdk-data-line={35}
\section{\mdline{35}2.\hspace*{0.5em}\mdline{35}Basics of Typesetting Code}\label{sec-basics-of-typesetting-code}%mdk%mdk

%mdk-data-line={37}
\noindent\mdline{37}Below is a snippet of JavaScript code%mdk
\begin{mdpre}%mdk
\noindent~~~{\mdcolor{navy}function}~hello()\{\\
~~~~~{\mdcolor{navy}return}~{\mdcolor{maroon}"}{\mdcolor{maroon}Hello~World!}{\mdcolor{maroon}"}\\
~~~\}%mdk
\end{mdpre}\noindent\mdline{43}Here is the Java code we mentioned in the first lecture:
\begin{mdpre}%mdk
\noindent~~{\mdcolor{navy}int}~total~=~{\mdcolor{purple}0};\\
~~{\mdcolor{navy}for}~({\mdcolor{navy}int}~i~=~{\mdcolor{purple}1};~i~\textless{}=~{\mdcolor{purple}10};~{\mdcolor{purple}1}++)\{\\
~~~~total~=~total~+~i;\\
~~\}%mdk
\end{mdpre}\noindent\mdline{50}And typing in Haskell is as easy:
\begin{mdpre}%mdk
\noindent~~f~{\mdcolor{teal}::}~{\mdcolor{teal}{}[}{\mdcolor{teal}Int}{\mdcolor{teal}]}~{\mdcolor{teal}-\textgreater{}}~{\mdcolor{teal}{}[}{\mdcolor{teal}Int}{\mdcolor{teal}]}\\
~~f{}[]~~~~~~~~{\mdcolor{navy}=}~{}[]\\
~~f~(x:xs)~{\mdcolor{navy}=}~f~ys~++~{}[x]~++~f~zs\\
~~~~{\mdcolor{navy}where}~ys~{\mdcolor{navy}=}~{}[a~{\mdcolor{navy}\textbar{}}~a~xs,~a~x]\\
~~~~zs~{\mdcolor{navy}=}~{}[b~{\mdcolor{navy}\textbar{}}~b~xs,~b~\textgreater{}~x]%mdk
\end{mdpre}
%mdk-data-line={59}
\section{\mdline{59}3.\hspace*{0.5em}\mdline{59}More Reading}\label{sec-more-reading}%mdk%mdk

%mdk-data-line={60}
\noindent\mdline{60}One of my favorite features we didn\mdline{60}'\mdline{60}t mention in class is the ability to add a photo; it seems really useful and looks really easy to do.
Another of my favorite is the ability to make tables/figures. Specifically the ability to make a matrix.%mdk

%mdk-data-line={63}
\section{\mdline{63}4.\hspace*{0.5em}\mdline{63}Hands-on}\label{sec-hands-on}%mdk%mdk

%mdk-data-line={68}
\begin{mdbmargintb}{4em}{}%mdk
\begin{mdflushright}%mdk
{\tiny\mdline{69}Created with~\href{https://www.madoko.net}{Madoko.net}.}%mdk
\end{mdflushright}%mdk
\end{mdbmargintb}%mdk%mdk


\end{document}
