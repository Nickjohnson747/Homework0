% MathMode: plain, MathRender: svg, MathDpi: 300, MathEmbedLimit: 524288, MathScale: 105, MathBaseline: 0, MathDocClass: [10pt]book, MathImgDir: math, MathLatex: xelatex, MathSvgUseFonts: False, MathSvgSharePaths: True, MathSvgPrecision: 3, Dvisvg: dvisvgm
\documentclass[10pt]{book}
% generated by Madoko, version 1.0.3
%mdk-data-line={1}
\newcommand\mdmathmode{plain}
\newcommand\mdmathrender{svg}
\usepackage[heading-base={2},section-num={false},bib-label={true}]{madoko2}


\begin{document}


\begin{mdSnippets}
%mdk-data-line={8}
%mdk-data-line={19}
\begin{mdInlineSnippet}[826676a6a5ad24552f0d5af1593434cc]%mdk
$E=mc^2$\end{mdInlineSnippet}%mdk
%mdk-data-line={21}
\begin{mdDisplaySnippet}[3ddeeecfa9b7b38221c376ed4d89a66b]%mdk
\[%mdk-data-line={22}
  e = \lim_{n\rightarrow\infty} (1+\frac1n)^n
\]%mdk
\end{mdDisplaySnippet}%mdk
%mdk-data-line={30}
\begin{mdDisplaySnippet}[d87c37accdb98b2b25c0c31e6b0eb368]%mdk
\[%mdk-data-line={31}
  a^{\phi(n)} \equiv1 \mod n
\]%mdk
\end{mdDisplaySnippet}%mdk
%mdk-data-line={72}

\end{mdSnippets}

\end{document}
